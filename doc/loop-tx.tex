\documentclass[10pt]{article}

\usepackage{amsmath}
\usepackage{algpseudocode}
\usepackage{graphicx}
\usepackage{hyperref}
\usepackage{enumerate}

\title{Loop Diagonalization}
\author{Vedant Kumar \\ \texttt{vsk@berkeley.edu}}

\begin{document}
\maketitle

\section{Abstract}

The eigenvalue equation $Mv = \lambda v$ is a powerful statement about
matrix and scalar multiplications. In the context of compiler optimization,
we can use it to transform linearizable loops which run in $O(n)$ steps into
matrix operations which run in $O(\log n)$ steps.  This paper defines loop
diagonalization (the process of rewriting linearizable loops as efficient
eigendecompositions), discusses the implementation of this optimization with
LLVM, and discusses the advantages and limitations of this method.

\section{Overview}

Roughly speaking, a linearizable loop may equally well be represented as a
matrix $M$. Let the loop's state variables reside in a vector $v$. Each
iteration of the loop effects the update $v' = Mv$.  

It follows inductively that if one iteration of a linearizable loop gives
$Mv$, $n$ iterations of the loop can be simulated by computing $M^n v$.
Performing this matrix exponentiation naively requires up to $O(nm^3)$
scalar multiplictaions.

Linear algebra comes to the rescue. Recall the equation $M = PDP^{-1}$.  The
column vectors of $P$ are the eigenvectors of $M$, and the diagonal matrix
$D$ contains the corresponding eigenvalues. We can find this
eigendecomposition for any square $M$ so long as $P$ is invertible.

It is known that $M^n = PD^nP^{-1}$, and that $D^n$ can be computed in
$O(m\log n)$ steps using the repeated squaring algorithm. With these facts
we can replace linearizable loops with fast matrix operations.

\section{Diagonalization}

The following definitions are useful: \\

\noindent
\textbf{Def. Linearizable Loop}: A tuple $(L, M, v)$ where $L$ is a list of
basic blocks with a preheader and one backedge, $M$ is a $m$ by $m$ square
matrix with $m$ linearly independent eigenvectors, and $v$ is a vector of
state variables. Let $v' = Mv$: the \emph{only} permissible instructions in
$L$ are the arithmetic operations which contribute to the goal of updating
$v$ to $v'$ with no other side effects, excepting the branch and jump
instructions required to construct a loop. \\

\noindent
\textbf{Def. Loop Diagonalization}: Given a program containing a linearizable
loop $(L, M, v)$, replace $L$ with a single basic block that computes $M^n
v$ using $PD^nP^{-1}v$. \\

Loop diagonalization transforms functions in $O(nm^3)$ into two matrix
multiplications and a diagonal matrix exponentiation, which is in $O(m^3 +
m\log n)$. This can result in an appreciable increase in program efficiency,
as will be shown.

\section{The Fibonacci Example}

Consider the iterative procedure for computing the $n$-th Fibonacci number:

\begin{center}
    \parbox{4cm}{
        \begin{algorithmic}
            \Function{fib}{n}
                \State $a\gets 1$
                \State $b\gets 1$
                \For{$i \in [2 ... n]$}
                    \State $tmp\gets a$
                    \State $a\gets b$
                    \State $b\gets tmp + b$
                \EndFor
                \State \Return b
            \EndFunction
        \end{algorithmic}
    }
\end{center}

The state vector is $v = (a, b)^T$. Initially, $v_0 = (1, 1)^T$. After each
iteration of the loop, $a' = b$ and $b' = a + b$. These linear combinations
are encoded by:

\begin{displaymath}
M = \begin{bmatrix}
        0 & 1 \\
        1 & 1
    \end{bmatrix}
\end{displaymath}

Let $\phi = \frac{1 + \sqrt{5}}{2}$. The eigenvalues of $M$ are $\phi$ and
$1-\phi$. The corresponding eigenvectors are $(1, \phi)^T$ and $(\phi,
-1)^T$. $M$ is diagonalizable because its eigenvalues are distinct and its
eigenvectors are linearly independent:

\begin{displaymath}
M = \begin{bmatrix}
    1 & \phi \\
    \phi & -1
    \end{bmatrix}
    \begin{bmatrix}
    \phi & 0 \\
    0 & 1-\phi
    \end{bmatrix}
    \begin{bmatrix}
    1 & \phi \\
    \phi & -1
    \end{bmatrix}^{-1}
\end{displaymath}

Given $M = PDP^{-1}$, $M^n = PD^nP^{-1}$:

\begin{align*}
M^n &= \begin{bmatrix}
    1 & \phi \\
    \phi & -1
    \end{bmatrix}
    \begin{bmatrix}
    \phi & 0 \\
    0 & 1-\phi
    \end{bmatrix}^n
    \begin{bmatrix}
    1 & \phi \\
    \phi & -1
    \end{bmatrix}^{-1} \\
    &= \begin{bmatrix}
    1 & \phi \\
    \phi & -1
    \end{bmatrix}
    \begin{bmatrix}
    \phi^n & 0 \\
    0 & (1-\phi)^n
    \end{bmatrix}
    \begin{bmatrix}
    1 & \phi \\
    \phi & -1
    \end{bmatrix}^{-1}
\end{align*}

Now the Fibonacci loop has been diagonalized:

\begin{center}
    \parbox{6cm}{
        \begin{algorithmic}
            \Function{fib}{n}
                \State $(a, b)^T \gets PD^{n-1}P^{-1}(1, 1)^T$
                \State \Return b
            \EndFunction
        \end{algorithmic}
    }
\end{center}

\section{Implementation}

I implemented automatic loop diagonalization using the LLVM compiler
infrastructure. This involved creating an instance of
\texttt{llvm::LoopPass} (\texttt{ADPass}) which can be loaded from a dynamic
library. \texttt{ADPass} determines if a loop is linearizable by filtering
out unsupported instructions. It then builds the loop matrix $M$ by
performing a depth-first search on the exit phi-nodes in the loop. This
search allows the pass to determine the coefficients of every linear
combination of $v$ in the loop, which are exactly the entries of $M$. Next,
the pass uses the \texttt{Eigen} library to find the eigendecomposition of
$M$. Finally the pass deletes the original loop, inserts newly generated
bitcode corresponding to the decomposition into the program, and rewires the
values flowing into the exit phi-nodes. The source code is available under a
non-restrictive free software license
\href{https://github.com/vedantk/auto-diagonalize}{here}.

\section{Testing}

To test the loop diagonalization algorithm, I created a test suite of
iterative processes written in C++ and compiled them at the highest
optimization levels available in LLVM 3.2 and clang++ on Linux x86-64. I
then created alternate versions of the binaries which were post-processed
with \texttt{ADPass}. \texttt{ADPass} was able to diagonalize basic loops
but depended upon pre-processing by the compiler to eliminate stack-spills
and canonicalize loops. It produced good results for all tested programs. I
did not attempt to analyze loops which I knew the pass cannot optimize.

Consider the iterative Fibonacci procedure discussed in a previous section.
I measured the time it took to call $fib(0)$, ..., $fib(200)$ 50,000 times
each. I first tried this with the normal binary and then compared its
performance to the auto-diagonalized binary. As expected, the normal program
(left) scales linearly with the size of the input, whereas the diagonalized
program (right) exhibits $O(\log n)$ performance. 

\begin{center}
\includegraphics[scale=0.42]{"fib-O3"}
\includegraphics[scale=0.42]{"fib-O3-diag"}
\end{center}

\section{Discussion}

Matrix decompositions have applications in many fields, but their use in
compiler loop optimizations appears to be novel \footnote{This might have
been true when it was written in 2012, but papers on abstract acceleration
of general linear loops have been published since then.}. It isn't clear if
this optimization pays for its implementation complexity.

Leaving the issue of practicality aside, there are important technical
limitations to \texttt{ADPass}. The most severe restriction is that function
calls, branches, and other instructions with unpredictable side effects
cannot occur within a linearizable loop. In general it is not possible to
lift this draconian restriction without solving undecidable problems such as
`when exactly is this branch taken?' and `does this function call return?'.
Another problem is that there is inherent numerical instability in IEEE 754
floating point numbers, which means that the effects of auto-diagonalization
are not always completely transparent.

With that said, we're left with a somewhat academic optimization that
produces some interesting results. It allows linearizable $O(nm^3)$
processes to run in $O(m^3 + m\log n)$, using the simple magic trick $Mv =
\lambda v$.

\section{References}

\begin{enumerate}[1.]
    \item LLVM, \url{http://llvm.org/}.
    \item Eigen, v3. Ga\"{e}l Guennebaud and Beno\^{i}t Jacob and others.
        \url{http://eigen.tuxfamily.org}.
\end{enumerate}

\end{document}
